\documentclass[11tp, a4paper]{article}
\usepackage[MeX]{polski}
\usepackage[left=2.5cm]{geometry}
\usepackage[utf8]{inputenc}
\usepackage{caption}
\usepackage{amssymb}
\usepackage{amsmath}
\usepackage{enumerate}
\usepackage{multirow}
\begin{document}
ułamek w tekscie : $ \frac{1}{x} $ \\
oto równanie : $$ c^{2}=a^{2}+b^{2} $$  %% środkuje \\
\linebreak
\linebreak

\begin{equation}
\frac{1}{x} 
\end{equation}
oto równianie:
\begin{equation}
c^{2}=a^{2}+b^{2} 
\end{equation}
indeks górny: $$ x^{y} \ e^{x} \ 2^{e} \ A^{2\times2} $$

index dolny: $$ x_y \ a_{ij} \ x_i $$ \\

indeks ten i ten: $$ x_i^{2} \ x_{i^2}^{k_j} $$
\linebreak

$$ \sqrt{\frac{2^2}{2_n}} \neq \sqrt[ \frac{1}{3}]{1+n} $$



$$ \sum \sum_{i=1}^{10}x_i \ \int \ \bigcap \ \bigcup \ \bigsqcup
\bigvee \ \bigwedge $$

$$ \int\int_{d}\ \mathrm{d} x \, \mathrm{d} y  $$

$$ \int\!\!\int_{d} \, \mathrm{d} x \, \mathrm{d} y$$

$$ {n \choose k}\qquad {x \atop y+2} $$

\end{document}